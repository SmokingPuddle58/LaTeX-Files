% Include other package you wish to use, if necessary
% The use of emoji package requires lualatex to compile. Please notice that.
\usepackage{tcolorbox}
\usepackage{ntheorem}
\usepackage{amsmath}
\usepackage{amssymb}
\usepackage{ulem}
\usepackage{graphicx}
\usepackage{centernot}
\usepackage{pgfplots}
\usepackage{tikz}
\usepackage{tikz-cd}
\usepackage{tabularx}
\usepackage{makecell}
\usepackage{mathtools}
% \usepackage{emoji}
\usepackage{nicematrix}
\usepackage{xcolor}
\usepackage{listings}

% Please keep this setting 
\pgfplotsset{compat=1.18}

% Hyperlink settings
\usepackage[hidelinks, linktocpage]{hyperref}
\hypersetup{
    linktoc=all
}

% Adjest margin here
\usepackage[margin=1in]{geometry}
\usepackage[parfill]{parskip}

% Remove this if you do not want your inline formula to be displayed
% like full-line formula
\everymath{\displaystyle}

% I forgot what this does
\makeatletter
\newtheoremstyle{MyNonumberplain}
  {\item[\theorem@headerfont\hskip\labelsep ##1\theorem@separator]}%
  {\item[\theorem@headerfont\hskip\labelsep ##3\theorem@separator]}%
\makeatother
\theoremstyle{MyNonumberplain}
\theorembodyfont{\upshape}

% Include new commands to help with your typing
\newcommand{\R}{\mathbb{R}}
\newcommand{\Q}{\mathbb{Q}}
\newcommand{\Z}{\mathbb{Z}}
\newcommand{\N}{\mathbb{N}}
\newcommand{\C}{\mathbb{C}}
\newcommand{\nin}{\not\in}
\newcommand{\p}{\phi}
\newcommand{\ve}{\varepsilon}
\newcommand{\ev}{\mathbb{E}}
\newcommand{\var}{\text{Var}}
\newcommand{\cov}{\text{Cov}}
\newcommand{\T}{^\intercal}
\newcommand{\der}{\operatorname{d\!}{}}
\newcommand{\evd}{\ev_{\mathcal{D}}}
\newcommand{\D}{\mathcal{D}}
\newcommand{\bias}{\text{Bias}}
\newcommand{\bt}[1]{\beta_{#1}}
\newcommand\ddfrac[2]{\frac{\displaystyle #1}{\displaystyle #2}}
\newcommand{\matindex}[1]{\mbox{\scriptsize#1}}% Matrix index
\newcommand{\inv}{^{-1}}
\newcommand{\pd}[2]{\frac{\partial {#1}}{\partial {#2}}}
\newcommand{\tr}{\text{tr}}
\newcommand{\dd}{\mathrm{d}}

% You may add different boxes in different colors. 
\newtcolorbox{prfbox}{colback=gray!10,colframe=black!70,boxrule=0pt,arc=0pt,boxsep=2pt,left=2pt,right=2pt,leftrule=0pt}
\newtcolorbox{thmbox}{colback=orange!25,colframe=orange!85,boxrule=0pt,arc=0pt,boxsep=2pt,left=2pt,right=2pt,leftrule=2.5pt}
\newtcolorbox{defbox}{colback=blue!5,colframe=blue!70,boxrule=0pt,arc=0pt,boxsep=2pt,left=2pt,right=2pt,leftrule=2.5pt}
\newtcolorbox{ansbox}{colback=gray!10,colframe=black!70,boxrule=0pt,arc=0pt,boxsep=2pt,left=2pt,right=2pt,leftrule=0pt}
\newtcolorbox{expbox}{colback=green!10,colframe=green!70,boxrule=0pt,arc=0pt,boxsep=2pt,left=2pt,right=2pt,leftrule=2.5pt}
\newtcolorbox{warnbox}{colback=red!15,colframe=red!70,boxrule=0pt,arc=0pt,boxsep=2pt,left=2pt,right=2pt,leftrule=2.5pt}
\newtcolorbox{notebox}{colback=magenta!15,colframe=magenta!70,boxrule=0pt,arc=0pt,boxsep=2pt,left=2pt,right=2pt,leftrule=2.5pt}

% You may add different type of keywords, and do note that non-English characters are supported if you wish.
\newtheorem{warning}{Warning}[section]
\newtheorem{remark}{Remark}[section]
\theoremstyle{break}
\newtheorem{theorem}{Theorem}[section]
\newtheorem{corollary}{Corollary}[theorem]
\newtheorem{proposition}{Proposition}[section]
\newtheorem{example}{Example}[section]
\newtheorem{lemma}[theorem]{Lemma}
\newtheorem{note}{Note}
\newtheorem*{question}{Question}
\newtheorem*{answer}{Answer}
\newtheorem{definition}{Definition}[section]
\theoremstyle{break}
\newtheorem*{proof}{Proof. }

% The footnote command
\renewcommand{\thempfootnote}{\textbf{\textcolor{red}\arabic{mpfootnote}}}

% Modify your title and author here.
\title{Titles}
\author{Author}
