%!TEX program = lualatex
\documentclass{article}
\usepackage{tcolorbox}
\usepackage{ntheorem}
\usepackage{amsmath}
\usepackage{amssymb}
\usepackage{ulem}
\usepackage{graphicx}
\usepackage{centernot}
\usepackage{tikz}
\usepackage{tikz-cd}
\usepackage{emoji}
\usepackage{tabularx}
\usepackage{pgfcore}

\usepackage[hidelinks, linktocpage]{hyperref}
\hypersetup{
    linktoc=all,
}

\usepackage[margin=1in]{geometry}
\usepackage[parfill]{parskip}

\everymath{\displaystyle}

\makeatletter
\newtheoremstyle{MyNonumberplain}%
  {\item[\theorem@headerfont\hskip\labelsep ##1\theorem@separator]}%
  {\item[\theorem@headerfont\hskip\labelsep ##3\theorem@separator]}%
\makeatother
\theoremstyle{MyNonumberplain}
\theorembodyfont{\upshape}

\theoremstyle{break}
\newtheorem*{proof}{Proof. }

\newcommand{\tmmathbf}[1]{\ensuremath{\boldsymbol{#1}}}
\newcommand{\R}{\mathbb{R}}
\newcommand{\Q}{\mathbb{Q}}
\newcommand{\Z}{\mathbb{Z}}
\newcommand{\N}{\mathbb{N}}
\newcommand{\C}{\mathbb{C}}
\newcommand{\cyclic}[1]{\langle #1 \rangle}
\newcommand{\nline}{\begin{tabular}{ll}&\\\end{tabular}}
\newcommand{\nin}{\not\in}
\newcommand{\p}{\phi}
\newcommand{\infixor}{\text{ or }}
\newcommand{\infixand}{\text{ and }}
\newcommand{\ord}[1]{\text{ord}(#1)}
\newcommand{\tmop}{\text}
\newcommand{\xequal}[1]{\stackrel{#1}{=}}
\newcommand{\tmscript}[1]{\text{\scriptsize{$#1$}}}



\newtcolorbox{prfbox}{colback=gray!10,colframe=black!70,boxrule=0pt,arc=0pt,boxsep=2pt,left=2pt,right=2pt,leftrule=0pt}
\newtcolorbox{thmbox}{colback=orange!25,colframe=orange!85,boxrule=0pt,arc=0pt,boxsep=2pt,left=2pt,right=2pt,leftrule=2.5pt}
\newtcolorbox{defbox}{colback=blue!5,colframe=blue!70,boxrule=0pt,arc=0pt,boxsep=2pt,left=2pt,right=2pt,leftrule=2.5pt}
\newtcolorbox{ansbox}{colback=gray!10,colframe=black!70,boxrule=0pt,arc=0pt,boxsep=2pt,left=2pt,right=2pt,leftrule=0pt}
\newtcolorbox{expbox}{colback=green!10,colframe=green!70,boxrule=0pt,arc=0pt,boxsep=2pt,left=2pt,right=2pt,leftrule=2.5pt}
\newtcolorbox{warnbox}{colback=red!15,colframe=red!70,boxrule=0pt,arc=0pt,boxsep=2pt,left=2pt,right=2pt,leftrule=2.5pt}

\newtheorem{warning}{Warning}[section]

\theoremstyle{break}
\newtheorem{theorem}{Theorem}[section]
\newtheorem{corollary}{Corollary}[theorem]
\newtheorem{proposition}{Proposition}[section]
\newtheorem{example}{Example}[section]
\newtheorem{lemma}[theorem]{Lemma}

\theoremstyle{break}
\theoremstyle{definition}
\theoremstyle{break}
\newtheorem{definition}{Definition}[section]

\title{Titles}
\author{Author}

\begin{document}

\maketitle

\begin{center}
    This work is licensed under CC BY-NC-SA 4.0
    
    Some other notes you may want to add
\end{center}


\newpage

    This is the introduction text for the template.

    \bigskip


\begin{thmbox}
    Theorems, Corollary, Lemma, Proposition
\end{thmbox}

\begin{defbox}
    Definitions
\end{defbox}

\begin{expbox}
    Examples
\end{expbox}

\begin{warnbox}
    Warnings
\end{warnbox}

\begin{prfbox}
    Proofs, Answers
\end{prfbox}

\begin{tabular}{ll}
    &\\
\end{tabular}

Some special symbols, notations and functions that will appear in this note:\bigskip

\begin{center}

    \begin{tabular}{|l|l|}
        \hline
        $\C$ & Set of complex numbers \\ \hline
        $\R$ & Set of real numbers \\ \hline
        $\Z$ & Set of integers \\ \hline
        $\Q$ & Set of rational numbers \\ \hline
    \end{tabular}
\end{center}
\begin{center}
    
    \begin{tabular}{|l|l|}
        \hline
        $\mathbb{S}^{*}$ & The set of $\mathbb{S}$ excluding 0 (Identity element for addition) \\ \hline
        ord$(a)$                                       & The order of the element $a$ in a group \\ \hline
        $|A|$                                          & Cardinality of set $A$ \\
        \hline                                                           
    \end{tabular}

\end{center}


\newpage

\tableofcontents

\newpage

% If you wish your section number begins from 1, remove / comment the line below
\setcounter{section}{-1}

\section{Sample section}

\end{document}

% Below are some boxes that may help with your document. They are not rendered during compilation.

% Definition
\begin{defbox}
    \begin{definition}[]

    \end{definition}
\end{defbox}

% Theorem (With a proof box included, remove if necessary)
\begin{thmbox}
    \begin{theorem}

    \end{theorem}
    \begin{prfbox}
        \begin{proof}

        \end{proof}
    \end{prfbox}
\end{thmbox}

% Example
\begin{expbox}
    \begin{example}

    \end{example}
\end{expbox}
