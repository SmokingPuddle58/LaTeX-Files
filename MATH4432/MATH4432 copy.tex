%!TEX program = lualatex
\documentclass{article}
\usepackage{tcolorbox}
\usepackage{ntheorem}
\usepackage{amsmath}
\usepackage{amssymb}
\usepackage{ulem}
\usepackage{graphicx}
\usepackage{centernot}
\usepackage{tikz}
\usepackage{tikz-cd}
\usepackage{tabularx}
\usepackage{makecell}
\usepackage{mathtools}
\usepackage[hidelinks, linktocpage]{hyperref}
\hypersetup{
    linktoc=all,
}

\usepackage[margin=1in]{geometry}
\usepackage[parfill]{parskip}

\everymath{\displaystyle}

\makeatletter
\newtheoremstyle{MyNonumberplain}%
  {\item[\theorem@headerfont\hskip\labelsep ##1\theorem@separator]}%
  {\item[\theorem@headerfont\hskip\labelsep ##3\theorem@separator]}%
\makeatother
\theoremstyle{MyNonumberplain}
\theorembodyfont{\upshape}

\theoremstyle{break}
\newtheorem*{proof}{Proof. }

\newcommand{\R}{\mathbb{R}}
\newcommand{\Q}{\mathbb{Q}}
\newcommand{\Z}{\mathbb{Z}}
\newcommand{\N}{\mathbb{N}}
\newcommand{\C}{\mathbb{C}}
\newcommand{\nin}{\not\in}
\newcommand{\p}{\phi}
\newcommand{\ev}{\mathbb{E}}
\newcommand{\var}{\text{Var}}
\newcommand{\der}{\operatorname{d\!}{}}
\newcommand{\evd}{\ev_{\mathcal{D}}}



\newtcolorbox{prfbox}{colback=gray!10,colframe=black!70,boxrule=0pt,arc=0pt,boxsep=2pt,left=2pt,right=2pt,leftrule=0pt}
\newtcolorbox{thmbox}{colback=orange!25,colframe=orange!85,boxrule=0pt,arc=0pt,boxsep=2pt,left=2pt,right=2pt,leftrule=2.5pt}
\newtcolorbox{defbox}{colback=blue!5,colframe=blue!70,boxrule=0pt,arc=0pt,boxsep=2pt,left=2pt,right=2pt,leftrule=2.5pt}
\newtcolorbox{ansbox}{colback=gray!10,colframe=black!70,boxrule=0pt,arc=0pt,boxsep=2pt,left=2pt,right=2pt,leftrule=0pt}
\newtcolorbox{expbox}{colback=green!10,colframe=green!70,boxrule=0pt,arc=0pt,boxsep=2pt,left=2pt,right=2pt,leftrule=2.5pt}
\newtcolorbox{warnbox}{colback=red!15,colframe=red!70,boxrule=0pt,arc=0pt,boxsep=2pt,left=2pt,right=2pt,leftrule=2.5pt}
\newtcolorbox{notebox}{colback=magenta!15,colframe=magenta!70,boxrule=0pt,arc=0pt,boxsep=2pt,left=2pt,right=2pt,leftrule=2.5pt}

\newtheorem{warning}{Warning}[section]
\newtheorem{remark}{Remark}[section]
\theoremstyle{break}
\newtheorem{theorem}{Theorem}[section]
\newtheorem{corollary}{Corollary}[theorem]
\newtheorem{proposition}{Proposition}[section]
\newtheorem{example}{Example}[section]
\newtheorem{lemma}[theorem]{Lemma}
\newtheorem{note}{Note}
\newtheorem*{answer}{Answer}


\theoremstyle{break}
\theoremstyle{break}
\newtheorem{definition}{Definition}[section]

\title{MATH4432 Homework 1 [Conceptual Parts]}
\author{Zhang Zhe 20866321}

\begin{document}

\maketitle

\newpage

\section*{Chapter 2}

1. For each of parts (a) through (d), indicate whether we would generally
expect the performance of a flexible statistical learning method to be
better or worse than an infexible method. Justify your answer.

(a) The sample size $n$ is extremely large, and the number of predictors $p$ is small.

(b) The number of predictors $p$ is extremely large, and the number
of observations $n$ is small.

(c) The relationship between the predictors and response is highly
non-linear.

(d) The variance of the error terms, i.e. $\sigma^2$ = Var($\varepsilon$), is extremely
high.

\begin{prfbox}
    \begin{answer}  
        (a) 
        \bigskip
        (b)

        \bigskip

        (c) The performance for flexible statistical learning method will in general be better, 
        it is because a flexible method is able to fit into more data points, whilst an inflexible method
        is unable to fit most of the data points, and this brings underfitting issue.

        \bigskip

        (d) If the variance of the error terms is high, it indicates that the sample contains a lot of noises,
        and performance of a flexible statistical learning method will be worse, since the model may fit into these
        unwanted noises.
    \end{answer} 
\end{prfbox}


3. We now revisit the bias-variance decomposition.

(a) Provide a sketch of typical (squared) bias, variance, training error, test error, and Bayes (or irreducible) error curves, on a single plot, as we go from less fexible statistical learning methods
towards more fexible approaches. The $x$-axis should represent
the amount of fexibility in the method, and the $y$-axis should
represent the values for each curve. There should be five curves.
Make sure to label each one.

(b) Explain why each of the five curves has the shape displayed in
part (a).

\end{document}

% Below are some boxes that may help with your document. They are not rendered during compilation.

% Definition
\begin{defbox}
    \begin{definition}[]

    \end{definition}
\end{defbox}

% Theorem (With a proof box included, remove if necessary)
\begin{thmbox}
    \begin{theorem}

    \end{theorem}
    \begin{prfbox}
        \begin{proof}
            
        \end{proof}
    \end{prfbox}
\end{thmbox}

% Example
\begin{expbox}
    \begin{example}

    \end{example}
\end{expbox}

% Warning / Remarks
\begin{warnbox}
    \begin{warning}

    \end{warning}
\end{warnbox}

\begin{warnbox}
    \begin{remark}

    \end{remark}
\end{warnbox}

% Notes
\begin{notebox}
    \begin{note}

    \end{note}
\end{notebox}